\documentclass[12pt]{article}
\usepackage[utf8]{inputenc}
\usepackage[spanish]{babel}
\usepackage{geometry}
\usepackage{listings}
\usepackage{xcolor}
\usepackage{graphicx}
\usepackage{hyperref}
\usepackage{helvet}
\renewcommand{\familydefault}{\sfdefault}

\geometry{a4paper, margin=2.5cm}

\definecolor{codegreen}{rgb}{0,0.6,0}
\definecolor{codegray}{rgb}{0.5,0.5,0.5}
\definecolor{codepurple}{rgb}{0.58,0,0.82}
\definecolor{backcolour}{rgb}{0.95,0.95,0.92}

\lstdefinestyle{mystyle}{
    backgroundcolor=\color{backcolour},   
    commentstyle=\color{codegreen},
    keywordstyle=\color{magenta},
    numberstyle=\tiny\color{codegray},
    stringstyle=\color{codepurple},
    basicstyle=\ttfamily\footnotesize,
    breakatwhitespace=false,         
    breaklines=true,                 
    captionpos=b,                    
    keepspaces=true,                 
    numbers=left,                    
    numbersep=5pt,                  
    showspaces=false,                
    showstringspaces=false,
    showtabs=false,                  
    tabsize=2
}

\lstset{style=mystyle}

\title{Dashboard Cafeteria\\Sistema de Reportes Analíticos con Next.js y PostgreSQL}
\author{Luis Emilio Jaras Sanchez \\ 5A - AWOS}
\date{Febrero 2026}

\begin{document}

\maketitle
\newpage

\tableofcontents
\newpage

\section{Introducción}
El presente proyecto implementa un dashboard de reportes analíticos para una cafetería del campus universitario. La solución utiliza Next.js 15 con TypeScript en el frontend y PostgreSQL 15 como sistema de base de datos, todo orquestado mediante Docker Compose.

\subsection{Objetivo}
Desarrollar una aplicación web que visualice reportes SQL obtenidos desde VIEWS en PostgreSQL, implementando seguridad real mediante un usuario de aplicación con permisos limitados.

\subsection{Alcance}
El sistema permite analizar:
\begin{itemize}
    \item Ventas diarias con métricas agregadas.
    \item Productos más vendidos mediante ranking.
    \item Inventario en riesgo por stock bajo.
    \item Valor de vida del cliente (LTV).
    \item Mix de métodos de pago.
\end{itemize}

\section{Arquitectura del Sistema}
\subsection{Stack Tecnológico}
El sistema se basa en una arquitectura moderna de tres capas:
\begin{itemize}
    \item \textbf{Frontend:} Next.js 15 (App Router) con Tailwind CSS para una interfaz reactiva.
    \item \textbf{Backend:} Next.js Server Components para la orquestación y lógica de negocio.
    \item \textbf{Base de Datos:} PostgreSQL 15, utilizando vistas y seguridad granular.
\end{itemize}

\subsection{Modelo de Datos}
El modelo relacional consta de 6 tablas:
\begin{enumerate}
    \item \texttt{categorias} - Categorías de productos.
    \item \texttt{productos} - Catálogo con stock.
    \item \texttt{clientes} - Clientes registrados.
    \item \texttt{ordenes} - Órdenes de compra.
    \item \texttt{detalle\_orden} - Detalle de ítems por orden.
    \item \texttt{metodos\_pago} - Pagos asociados a las órdenes.
\end{enumerate}

\subsection{Seguridad}
Se implementó el principio de mínimo privilegio creando un usuario \texttt{app\_user} que solo tiene permisos de lectura (\texttt{SELECT}) sobre las vistas analíticas, protegiendo las tablas base de accesos no autorizados.

\section{Implementación de VIEWS}

\subsection{vw\_ventas\_diarias - Ventas Diarias}
\textbf{Características:} Agregados (SUM, COUNT, AVG), ordenación por fecha.
\begin{lstlisting}[language=SQL, caption=Vista de ventas diarias]
CREATE OR REPLACE VIEW vw_ventas_diarias AS
SELECT 
    DATE(o.fecha_creacion) as fecha_venta,
    COUNT(DISTINCT o.id) as total_tickets,
    SUM(det.cantidad * det.precio_unitario) as venta_total,
    AVG(det.cantidad * det.precio_unitario) as ticket_promedio
FROM ordenes o
JOIN detalle_orden det ON o.id = det.id_orden
WHERE o.estado = 'completado'
GROUP BY DATE(o.fecha_creacion)
ORDER BY fecha_venta DESC;
\end{lstlisting}

\subsection{vw\_productos\_top - Ranking de Productos}
\textbf{Características:} Uso de Window Function \texttt{RANK()}.
\begin{lstlisting}[language=SQL, caption=Ranking con Window Function]
CREATE OR REPLACE VIEW vw_productos_top AS
SELECT 
    p.id as id_producto,
    p.nombre as nombre_producto,
    c.nombre as categoria,
    SUM(det.cantidad) as unidades_vendidas,
    SUM(det.cantidad * det.precio_unitario) as venta_total,
    RANK() OVER (ORDER BY SUM(det.cantidad * det.precio_unitario) DESC) as ranking_ventas
FROM productos p
JOIN detalle_orden det ON p.id = det.id_producto
JOIN ordenes o ON det.id_orden = o.id
JOIN categorias c ON p.id_categoria = c.id
WHERE o.estado = 'completado'
GROUP BY p.id, p.nombre, c.nombre;
\end{lstlisting}

\newpage
\subsection{vw\_riesgo\_inventario - Riesgo de Inventario}
\textbf{Características:} Expresión \texttt{CASE} para clasificación.
\begin{lstlisting}[language=SQL, caption=Clasificación con CASE]
CREATE OR REPLACE VIEW vw_riesgo_inventario AS
SELECT 
    p.id as id_producto,
    p.nombre as nombre_producto,
    c.nombre as categoria,
    p.stock as stock_actual,
    20 as umbral_reorden,
    CASE 
        WHEN p.stock = 0 THEN 'Critico'
        WHEN p.stock < 10 THEN 'Alto'
        WHEN p.stock < 20 THEN 'Bajo'
        ELSE 'Normal'
    END as nivel_riesgo
FROM productos p
JOIN categorias c ON p.id_categoria = c.id
WHERE p.stock < 20
ORDER BY p.stock ASC;
\end{lstlisting}

\subsection{vw\_valor\_cliente - Valor del Cliente}
\textbf{Características:} Cláusula \texttt{HAVING} y \texttt{COALESCE}.
\begin{lstlisting}[language=SQL, caption=LTV con HAVING]
CREATE OR REPLACE VIEW vw_valor_cliente AS
SELECT 
    c.id as id_cliente,
    c.nombre as nombre_cliente,
    c.email,
    COUNT(o.id) as total_ordenes,
    COALESCE(SUM(mp.monto), 0) as valor_vida,
    COALESCE(AVG(mp.monto), 0) as valor_promedio_orden
FROM clientes c
LEFT JOIN ordenes o ON c.id = o.id_cliente
LEFT JOIN metodos_pago mp ON o.id = mp.id_orden
GROUP BY c.id, c.nombre, c.email
HAVING COUNT(o.id) > 0
ORDER BY valor_vida DESC;
\end{lstlisting}

\newpage
\subsection{vw\_mix\_pagos - Mix de Pagos}
\textbf{Características:} Uso de CTE (\textit{Common Table Expression}).
\begin{lstlisting}[language=SQL, caption=Vista con CTE]
CREATE OR REPLACE VIEW vw_mix_pagos AS
WITH TotalVentas AS (
    SELECT SUM(monto) as gran_total 
    FROM metodos_pago
)
SELECT 
    mp.metodo as metodo_pago,
    COUNT(mp.id) as conteo_transacciones,
    SUM(mp.monto) as monto_total,
    ROUND((SUM(mp.monto) / (SELECT gran_total FROM TotalVentas)) * 100, 2) as porcentaje_participacion
FROM metodos_pago mp
GROUP BY mp.metodo
HAVING SUM(mp.monto) > 0
ORDER BY monto_total DESC;
\end{lstlisting}

\section{Optimización con Índices}
Se implementaron índices estratégicos para mejorar el rendimiento de las consultas de agregación:
\begin{itemize}
    \item \texttt{idx\_ordenes\_fecha} en \texttt{ordenes(fecha\_creacion)}.
    \item \texttt{idx\_productos\_categoria} en \texttt{productos(id\_categoria)}.
    \item \texttt{idx\_detalle\_producto} en \texttt{detalle\_orden(id\_producto)}.
\end{itemize}

\section{Capturas de Pantalla}
\textit{[Nota: Insertar imágenes en los siguientes placeholders]}

\subsection{Dashboard Principal}
\subsection{Reporte de Ventas Diarias}
\subsection{Reporte de Productos Top}
\subsection{Reporte de Inventario}
\subsection{Reporte de Clientes VIP}

\section{Despliegue con Docker}
\subsection{Comando de Ejecución}
Para iniciar la infraestructura completa, se utiliza el siguiente comando:
\begin{verbatim}
docker-compose up --build
\end{verbatim}
La aplicación queda disponible en \url{http://localhost:3001}.

\section{Cumplimiento de Requisitos}
El sistema cumple satisfactoriamente con todos los requisitos de la Práctica C1: conexión a base de datos, creación de vistas complejas, interfaz de reportes interactiva y políticas de seguridad granular.

\section{Conclusiones}
El proyecto implementa exitosamente un sistema de reportes analíticos con arquitectura moderna, segura y escalable, logrando una integración fluida entre PostgreSQL y Next.js.

\subsection{Logros Principales}
\begin{enumerate}
    \item Sistema completamente containerizado.
    \item 5 VIEWS SQL con características avanzadas (Window Functions, CTE, CASE).
    \item Seguridad mediante mínimo privilegio.
    \item Interfaz responsiva y profesional.
\end{enumerate}

\newpage
\section*{Fuentes de Consulta}
\begin{enumerate}
    \item Google DeepMind. (2026). "Gemini - Advanced Agentic Coding and Logical Reasoning".
    \item Next.js Documentation. (2024). \textit{App Router}. Vercel, Inc.
    \item PostgreSQL Global Development Group. (2024). \textit{PostgreSQL 15 Documentation}.
    \item Docker Inc. (2024). \textit{Docker Compose Documentation}.
    \item Vercel. (2024). \textit{Data Fetching with Server Components}.
\end{enumerate}

\end{document}
